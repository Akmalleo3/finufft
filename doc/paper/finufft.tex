\documentclass[10pt]{article}
\textwidth 6.5in
\oddsidemargin=0in
\evensidemargin=0in

\usepackage{graphicx,bm,amssymb,amsmath,amsthm}

% general macros...
\newcommand{\bi}{\begin{itemize}}
\newcommand{\ei}{\end{itemize}}
\newcommand{\ben}{\begin{enumerate}}
\newcommand{\een}{\end{enumerate}}
\newcommand{\be}{\begin{equation}}
\newcommand{\ee}{\end{equation}}
\newcommand{\bea}{\begin{eqnarray}} 
\newcommand{\eea}{\end{eqnarray}}
\newcommand{\ba}{\begin{align}} 
\newcommand{\ea}{\end{align}}
\newcommand{\bse}{\begin{subequations}} 
\newcommand{\ese}{\end{subequations}}
\newcommand{\bc}{\begin{center}}
\newcommand{\ec}{\end{center}}
\newcommand{\bfi}{\begin{figure}}
\newcommand{\efi}{\end{figure}}
\newcommand{\ca}[2]{\caption{#1 \label{#2}}}
\newcommand{\ig}[2]{\includegraphics[#1]{#2}}
\newcommand{\tbox}[1]{{\mbox{\tiny #1}}}
\newcommand{\mbf}[1]{{\mathbf #1}}
\newcommand{\half}{\mbox{\small $\frac{1}{2}$}}
\newcommand{\vt}[2]{\left[\begin{array}{r}#1\\#2\end{array}\right]} % 2-col-vec
\newcommand{\mt}[4]{\left[\begin{array}{rr}#1&#2\\#3&#4\end{array}\right]} % 2x2
\newcommand{\RR}{\mathbb{R}}
\newcommand{\eps}{\varepsilon}
\newcommand{\bigO}{{\mathcal O}}
\DeclareMathOperator{\re}{Re}
\DeclareMathOperator{\im}{Im}
\newtheorem{thm}{Theorem}
\newtheorem{cnj}[thm]{Conjecture}
\newtheorem{lem}[thm]{Lemma}
\newtheorem{cor}[thm]{Corollary}
\newtheorem{pro}[thm]{Proposition}
\newtheorem{rmk}[thm]{Remark}
% this work...
\newcommand{\xx}{\mbf{x}}
\newcommand{\sss}{\mbf{s}}
\newcommand{\yy}{\mbf{y}}
\newcommand{\kk}{\mbf{k}}
\newcommand{\KK}{{\mathcal K}}
\newcommand{\NU}{{non-uniform}}
\newcommand{\U}{{uniform}}
\newcommand{\KB}{Kaiser--Bessel}
\newcommand{\wid}{\beta}               % exp sqrt width param


\begin{document}
\title{FINUFFT: A lightweight non-uniform fast Fourier transform library}
\author{Alex Barnett and Jeremy Magland}
\date{\today}
\maketitle
\begin{abstract}
  Computation of Fourier transforms from data lying at arbitrary
  off-grid locations has many applications ranging from medical
  imaging to astronomy to fast algorithms for PDEs.  We present a
  software library for computing the non-uniform FFT (NFFT or NUFFT),
  of types 1 (\NU\ to \U), 2 (\U| to \NU), and
  3 (\NU\ to \NU), each in dimensions 1, 2, and 3.  Its
  main features are: a simple new kernel allowing faster on-the-fly
  spreading and interpolation from a regular grid, the use of
  quadrature rather than an analytic formula for the kernel Fourier
  transform, multi-threading for efficient use of shared-memory
  machines, simple calling interfaces matching those of the CMCL
  NUFFT library, and bindings to MATLAB/octave and python.  We
  compare performance of the library to existing libraries; typically
  we match the runtime of the NFFT library of Potts et al.\ but
  without the need for a precomputation phase.
\end{abstract}




\section{Introduction}

The computational task addressed
is to evaluate the following exponential sums to a requested precision,
in optimal time.
The type-1 NUFFT (also known as the adjoint NDFT \cite{usingnfft})
in dimension $d=1,2$ or 3
evaluates the Fourier series expansion for a set of
$M$ point sources at arbitrary locations $\xx_j$ in $[-\pi,\pi)^d$ with
strengths $f_j\in\mathbb{C}$,  $j=1,\dots,M$.
The outputs are the Fourier modes with integer indices lying in
the set
\be
\KK_{N_1,\dots,N_d} :=  \left[-\frac{N_1}{2},\frac{N_1-1}{2}\right]
\times \dots \times \left[-\frac{N_d}{2},\frac{N_d-1}{2}\right]
~,
\label{KK}
\ee
which is an interval of $N_1$ indices in 1D, a rectangle of $N_1N_2$
index pairs in 2D, or a cuboid of $N_1N_2N_3$ index triplets in 3D.%
\footnote{Note that these counts apply whether each $N_i$ is even or odd.
  For instance for $N_1=10$ in 1D, the set is $\{-5,-4,\dots,4\}$
  whereas for $N_1=11$, the set is $\{-5,-4,\dots,5\}$.
}
We will let $N=N_1\dots N_d$ denote the total number of output values.
We do not address $d>3$ here.
Following the normalization in \cite{dutt,nufft}, then,
\be
F(\kk) := \frac{1}{M} \sum_{j=1}^M f_j e^{i \kk\cdot \xx_j}
\qquad \mbox{for } \kk \in \KK_{N_1,\dots,N_d}
\hspace{1in} \mbox{(Type-1, or \NU\ to \U\ transform)}
~.
\label{1}
\ee
The naive evaluation of $F$ at all indices $\kk$ requires $\bigO(NM)$
exponential evaluations, which is prohibitive
in many applications.
However, it is well known that by resampling onto a regular
grid and using the fast Fourier transform (FFT),
the work can be reduced to $\bigO(M |\log\eps|^d + N \log N)$,
where $\eps$ is the requested precision.
Here the term $|\log\eps|^d$ is the number of points on the regular
grid that a single source affects through the spreading kernel.
GIVE REF.
Our library also exploits this idea.

The type-2 transform (or NDFT \cite{usingnfft})
is (up to a normalization factor) the adjoint of the
type-1, and evaluates a Fourier series with given coefficients
$F(\kk)$, $\kk\in\KK$ at an arbitrary set of target points
$\xx_j$, $j=1,\ldots,M$, which may be taken to be in $[-\pi,pi)^d$.
  That is,
  \be
  f_j := \sum_{\kk\in\KK_{N_1,\dots,N_d}} F(\kk) e^{i \kk\cdot \xx_j},
  \qquad j=1,\dots, M
\hspace{1in} \mbox{(Type-2, or \U\ to \NU\ transform)}
~.
\label{2}
\ee
Finally, the type-3 transform
\cite{nufft3} (or NNFFT \cite{usingnfft})
may be interpreted as evaluating the
Fourier transform of a set of sources at arbitrary locations $\xx_j$
in $\RR^d$
with strengths $f_j$, $j=1,\dots, M$, at the arbitrary target frequencies
$\sss_k$ in $\RR^d$, $k=1,\dots, N$. Note that now $k$ is a plain integer
index.
That is,
\be
F_k := \sum_{j=1}^M f_j e^{i \sss_k \cdot \xx_j}
  \qquad k=1,\dots, N
\hspace{1in} \mbox{(Type-3, or \NU\ to \NU\ transform)}
~.
\label{3}
\ee








Applications.

MRI.
CT.
Spectral interpolation from off-grid data.

Quadrature approximation of Fourier transforms, e.g.\
in image reconstruction \cite{cryo}.
Computation of spatially-periodic solutions to elliptic
PDEs via Ewald summation
\cite{lindbo11}.
Computation of history-dependent part for boundary-integral solvers
for the heat equation.%\cite{strain}.



Outline of algorithm:
EG type 1.
spreading to U grid using a kernel $\phi(\xx)$.
FFT.
Final correction of the Fourier modes by dividing by
$\hat\phi(k)$.


Existing implementations.

The first rigorous estimates of the type-1 and type-2 transforms
using the truncated Gaussian kernel was given in \cite{dutt}.

The CMCL NUFFT package \cite{cmcl,nufft,nufft3} uses truncated Gaussians.



The NFFT package \cite{nfft} from Chemnitz
includes the Kaiser--Bessel kernel,
which
achieves roughly twice the number of digits of precision
compare to the truncated Gaussian, for the same kernel width.

Fessler and Sutton \cite{fessler} use optimization in the space of
interpolation weights, enabling a slight increase
of around 1/3 of a digit in accuracy
over \KB in 1D. However, they conclude that
``the Kaiser-Bessel interpolator, with suitably
optimized parameters, represents a very reasonable com-
promise between accuracy and simplicity.''

In the below we present a kernel with performance essentially
equal to \KB, but which is simpler and faster to evaluate numerically.
In 1D, in rescaled spatial units, this kernel is
\be
\phi(z) = \phi_\beta(z) :=
\left\{\begin{array}{ll}
e^{\beta \sqrt{1-z^2}-1}, & |z|\le 1\\
0, & \mbox{otherwise}
\end{array}
\right.
~,
\label{ES}
\ee
where $\wid>0$ is a width parameter that depends on the kernel width
measured in uniform grid points.
In higher dimensions we use the product of 1D kernels.
We call this the ``exponential square-root'' (ES) kernel.



The features of our implementation include:
\bi
\item A simpler kernel with accuracy very similar to \KB\ but faster evaluation.
\item Use of quadrature rather than an analytic formula to evaluate the kernel Fourier transform needed for the
  correction (deconvolution) phase.
\item parallel operation (spreading and deconvolution) on shared-memory machines.
  \item use of the multi-threaded FFTW3 library for FFTs.
  \item compilation options such as single-precision (to reduce RAM footprint)
    and/or single-threaded.
  \item Bindings to MATLAB, octave, and python.
\ei

Unlike the authors of some other libraries,
we take the philosophy that the user calls the library to approximate the
exponential sums \eqref{1}--\eqref{3} to the requested precision.
Thus, the user does not have direct control over the type and width of
the spreading kernel; indeed, it would be confusing to have such control.
Rather, once the requested precision is given, such decisions
are made ``under the hood'' by the library.


\section{Algorithms}

\subsection{Type 1}

\subsection{Type 2}


\subsection{Type 3}

Can break it by choosing $XK$ huge, even with 2 input and 2 output pts.




\section{Error analysis}

FT defns.

\subsection{Error in the type-2 transform with general kernel}

see \cite[Sec.~V.B]{fessler}.




\subsection{Esimates on the Fourier transform of the new kernel}

Discuss relation of kernel \eqref{ES} to \KB.



scaled to width.

$w=1,2,\ldots$
sets the support of the kernel in units of the uniform grid point spacing, and





\section{Conclusion}


% BBBBBBBBBBBBBBBBBBBBBBBBBBBBBBBBBBBBBBBBBBBBBBBBBBBBBBBBBBBBBBBBBBBBBBBBBBBB
\bibliographystyle{abbrv}
\bibliography{alex}
\end{document}
